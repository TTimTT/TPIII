\documentclass[a4paper,12pt,oneside]{article}

\usepackage{graphicx}
\usepackage{verbatim}
\usepackage{amsmath}
\usepackage[english]{babel}
\usepackage[colorlinks,bookmarks=false,linkcolor=blue,urlcolor=blue]{hyperref}
\usepackage{booktabs}

\paperheight=297mm
\paperwidth=210mm

\setlength{\textheight}{235mm}
\setlength{\topmargin}{-1.2cm}
%\setlength{\footskip}{5mm}
\setlength{\textwidth}{15cm}
\setlength{\oddsidemargin}{0.56cm}
\setlength{\evensidemargin}{0.56cm}

\pagestyle{plain}


\def \be {\begin{equation}}
\def \ee {\end{equation}}
\def \dd  {{\rm d}}

\newcommand{\mail}[1]{{\href{mailto:#1}{#1}}}
\newcommand{\ftplink}[1]{{\href{ftp://#1}{#1}}}


\begin{document}

\title{Chaos Déterministe}
\author{Laurent Rohrbasser \& Tim Tuuva}

\maketitle
\tableofcontents
\baselineskip=16pt
\parindent=15pt
\parskip=5pt

\begin{abstract}
%Résumé de l'expérience, on fait des tps sur le chaos, rappeler vite fait 
%dire le but de ces manips, qu'est ce qu'on veut?
\end{abstract}

\section{Introduction}
%expliqué les enjeux théoriques sur le chaos.
%domaine très théoriques
%peu de compréhension sur le sujet ( pourquoi???)
En physique, il est souvent intéressant 


\section{Dispositif}
%Description du dispositif expérimental
%Tim's duty


\section{Résultats}
%Exposition des graphes avec les légendes et un peu de texte explicatifs sur comment on les a obtenus.

\begin{figure}[h!]
  \begin{center}
  %\includegraphics[width=0.8\linewidth,angle=0]{}
  \caption{} \label{fig:}
  \end{center}
\end{figure}

\section{Discussion}%le plus important

\subsection{Interprétation}
%Qu'est ce que les résultats nous permettent de conclure?
%Est ce que cela nous aide pour nôtre but?

\subsection{Discussions}
%Validité de nos résultats
%S'il y a des erreurs, d'ou viennent elle!
%Avons nous atteind les buts du TP?
\section{Conclusion}

%Résumer du rapport
%Ouverture





%Reference
\begin{thebibliography}{99}
\end{thebibliography}

\end{document}
