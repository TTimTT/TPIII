\documentclass[a4paper,12pt,oneside]{article}

\usepackage{graphicx}
\usepackage{verbatim}
\usepackage{amsmath}
\usepackage[english]{babel}
\usepackage[colorlinks,bookmarks=false,linkcolor=blue,urlcolor=blue]{hyperref}
\usepackage{booktabs}

\paperheight=297mm
\paperwidth=210mm

\setlength{\textheight}{235mm}
\setlength{\topmargin}{-1.2cm}
%\setlength{\footskip}{5mm}
\setlength{\textwidth}{15cm}
\setlength{\oddsidemargin}{0.56cm}
\setlength{\evensidemargin}{0.56cm}

\pagestyle{plain}


\def \be {\begin{equation}}
\def \ee {\end{equation}}
\def \dd  {{\rm d}}

\newcommand{\mail}[1]{{\href{mailto:#1}{#1}}}
\newcommand{\ftplink}[1]{{\href{ftp://#1}{#1}}}


\begin{document}

\title{Chaos Déterministe}
\author{Laurent Rohrbasser \& Tim Tuuva}

\maketitle
\tableofcontents
\baselineskip=16pt
\parindent=15pt
\parskip=5pt

\begin{abstract}
%Résumé de l'expérience, on fait des tps sur le chaos, rappeler vite fait 
%dire le but de ces manips, qu'est ce qu'on veut?
\end{abstract}

\section{Introduction}




1:
2:
3: Zn + O2






[x] A0 dépot 20° (No O2)
[x] Ax dépot 20° (No O2) + recuit 400°

[x] A1 dépot 20°
[x] A2 dépot 20° + recuit 400°
[x] A3 dépot 20° + recuit 600°

[x] B1 dépot 200°
[x] B2 dépot 200° + recuit 600°

[x] C1 dépot 400°
[x] C2 dépot 400° + recuit 600°



résistivité
[x]A0
[ ]Ax
[x]A1
[ ]A2
[ ]A3
[ ]B1
[ ]B2
[x]C1
[ ]C2

épaisseur (profilomètre)
[x]A0
[!]Ax
[x]A1
[!]A2
[!]A3
[x]B1
[!]B2
[x]C1
[!]C2

concentration charge par effet Hall
[?]A0
[?]Ax
[?]A1
[?]A2
[?]A3
[?]B1
[?]B2
[?]C1
[?]C2

Spectre
[x]A0
[x]Ax
[x]A1
[ ]A2
[x]A3
[x]B1
[x]B2
[x]C1
[x]C2

Diffraction 
[ ]A0
[ ]Ax
[ ]A1
[ ]A2
[ ]A3
[ ]B1
[ ]B2
[ ]C1
[ ]C2


trucs marrants a traiter :

- effet de la presence de O dans l'enceinte
- ration Zn - O 
--- affinite electronique


- la temperature dans l'enceinte augmente avec le plasma

- expliquer la reformation des microcristaus en un gros cristal a haute temperature
--- pourquoi les gros cristaux grandissent et les petits raptississent ?

- pourquoi depot a chaud change
--- energie de la plaque de verre transmise dans les particules de Zn, ainsi les Zn regagnent de l'energie et peuvent se mettre dans des endroits ou leur potentiel est plus faible.


\section{Dispositif}

\section{Résultats}
%Exposition des graphes avec les légendes et un peu de texte explicatifs sur comment on les a obtenus.

\begin{figure}[h!]
  \begin{center}
  %\includegraphics[width=0.8\linewidth,angle=0]{}
  \caption{} \label{fig:}
  \end{center}
\end{figure}

\section{Discussion}%le plus important

\subsection{Interprétation}
%Qu'est ce que les résultats nous permettent de conclure?
%Est ce que cela nous aide pour nôtre but?

\subsection{Discussions}
%Validité de nos résultats
%S'il y a des erreurs, d'ou viennent elle!
%Avons nous atteind les buts du TP?
\section{Conclusion}

%Résumer du rapport
%Ouverture





%Reference
\begin{thebibliography}{99}
\end{thebibliography}

\end{document}
