\documentclass[a4paper,12pt,oneside]{article}

\usepackage{graphicx}
\usepackage{verbatim}
\usepackage{amsmath}
\usepackage[english]{babel}
\usepackage[colorlinks,bookmarks=false,linkcolor=blue,urlcolor=blue]{hyperref}
\usepackage{booktabs}

\paperheight=297mm
\paperwidth=210mm

\setlength{\textheight}{235mm}
\setlength{\topmargin}{-1.2cm}
%\setlength{\footskip}{5mm}
\setlength{\textwidth}{15cm}
\setlength{\oddsidemargin}{0.56cm}
\setlength{\evensidemargin}{0.56cm}

\pagestyle{plain}


\def \be {\begin{equation}}
\def \ee {\end{equation}}
\def \dd  {{\rm d}}

\newcommand{\mail}[1]{{\href{mailto:#1}{#1}}}
\newcommand{\ftplink}[1]{{\href{ftp://#1}{#1}}}


\begin{document}

\title{}
\author{Laurent Rohrbasser \& Tim Tuuva}

\maketitle
\tableofcontents
\baselineskip=16pt
\parindent=15pt
\parskip=5pt

% \begin{abstract}
% THIS IS SO ABSTRACT 
% \end{abstract}


\section{Cheni}


différents pics :

courbe de balayage température.


courbe de balayage freq.
bruit de fond :
limite quand a la participation des trucs de fond





% \begin{figure}[h!]
%   \begin{center}
%   %\includegraphics[width=0.8\linewidth,angle=0]{}
%   \caption{} \label{fig:}
%   \end{center}
% \end{figure}

% \section{Introduction}

% \section{Dispositif}

% \section{Résultats}

% \section{Discussion}%le plus important

% \subsection{Interprétation}

% \section{Conclusion}





%Reference
% \begin{thebibliography}{99}
% \end{thebibliography}

\begin{table}[ht]
   \centering
   \begin{tabular}{|c|c|c|c|c|}
	  \hline
      Scan & $ln(\tau_0)$ & $\frac{E_a}{k_b}$ & $\tau_0$ [\rm{$s$}] & $E_a$ [J]\\
      \hline
      Fr\'equence & 0.0352 & -11695.4 & $4.837e^{-13}$ & $1.613e^{-19}$ \\
      Temp\'erature & 0.0332 & 12082.6 & $8.357e^{-14}$ & $1.667e^{-19}$ \\
      Refroidissement & 0.0278 & 15392.5 & $2.694e^{-16}$ & $2.124e^{-19}$ \\
      \hline
   \end{tabular}
   \caption{Tableau r\'ecapitulatif des r\'esultats par lecture du diagramme Arrhenius.}\label{tab:XXX}
\end{table}


\begin{table}[ht]
  \centering
   \begin{tabular}{|p{2cm}|c|}
	  \hline
      Fr\'equence d'excitation & ${E_a}'$ [J]\\
      \hline
      \centering 1.0 & $1.197e^{-19}$ \\
      \centering 2.0 & $1.178e^{-19}$ \\
      \centering 3.0 & $1.180e^{-19}$ \\
      \centering 5.0 & $1.214e^{-19}$ \\
      \centering 6.0 & $1.229e^{-19}$ \\
      \centering 7.0 & $1.197e^{-19}$ \\
      \centering 8.0 & $1.213e^{-19}$ \\
      \centering 9.0 & $1.178e^{-19}$ \\
      \centering 10.0 & $1.227e^{-19}$ \\
      \hline
   \end{tabular}
   \caption{Tableau r\'ecapitulatif des ${E_a}'$, par mesure de la demi-hauteur.}\label{tab:YYY}
\end{table}


\end{document}
