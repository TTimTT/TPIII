\documentclass[a4paper,12pt,oneside]{article}

\usepackage{subfig}
\usepackage{graphicx}
\usepackage{verbatim}
\usepackage{amsmath}
\usepackage[english]{babel}
\usepackage[utf8]{inputenc} 
\usepackage[colorlinks,bookmarks=false,linkcolor=blue,urlcolor=blue]{hyperref}
\usepackage{booktabs}

\paperheight=297mm
\paperwidth=210mm

\setlength{\textheight}{235mm}
\setlength{\topmargin}{-1.2cm}
%\setlength{\footskip}{5mm}
\setlength{\textwidth}{15cm}
\setlength{\oddsidemargin}{0.56cm}
\setlength{\evensidemargin}{0.56cm}

\pagestyle{plain}


\def \be {\begin{equation}}
\def \ee {\end{equation}}
\def \dd  {{\rm d}}

\newcommand{\mail}[1]{{\href{mailto:#1}{#1}}}
\newcommand{\ftplink}[1]{{\href{ftp://#1}{#1}}}


\begin{document}

\title{}
\author{Laurent Rohrbasser \& Tim Tuuva}

\maketitle
\tableofcontents
\baselineskip=16pt
\parindent=15pt
\parskip=5pt

\begin{abstract}
%Résumé de l'expérience, on fait des tps sur le chaos, rappeler vite fait 
%dire le but de ces manips, qu'est ce qu'on veut?
\end{abstract}

\section{Introduction}
Nuclear physic is one of the most famous and important field in physics, the understanding of these phenomena is a requirement for any physicist. It is a link to the deep matter complexity and can be used in different fields. In this experiments we introduce ourselves to nuclear physics and the radioactive decays. First of all, by measuring different desintegration spectrum, a calibration of the instrument is made. Then the study of nuclear phenomena statistic is done, after that the attenuation of gamma rays is studied using lead and aluminium plate and finally coincidence phenomena will be used to aged a radioactive source.

\section{Spectrometry $\gamma$}
\subsection{Theory}
Radioactive sources generates one or multiple gamma photons of various 
energy level. However, these photons cannot directly interact with the
 detector by a photoelectric effect due to their high energy level. 

This is why, sodium iodide cristal is used, thus to make the 
photons react with the cristal which generates electrons, this 
process generates new photons which will interact with electrons. 
This cristal realises, therefore, a chain reaction that amplifies the 
number of photons, which are less powerful than the original one, 
thus it can be detected on the photocathode. The measured tension 
depends on the energy level of the initial gamma photons.

There are several kind of phenomena that can happen on the gamma photons.
\subsubsection{Photoelectric effect}
This effect occurs when a photon has enough energy to free an electron from its atom. The kinetic energy of the ejected electron is given by:

$$T=E_p - E_l$$

, where $E_l$ is the link energy between the electron and the atom.

\subsubsection{Compton effect}
This effect occurs when a photon gives a part of its momentum to an 
electron as they interact. The electron will follow a direction given 
by the angle $\phi$ and a new photon will be emitted with an angle $\theta$.
The kinetic energy of the electron is given by:

$$T=\frac{E_p}{1+\frac{m_0c^2}{E_p(1-\cos(\theta))}}$$

The Compton effect creates a flat pattern on the energy spectrum, it ends at $E_{max}=T$,

One can notice that if the angle of scattering is $\theta=180^{\circ}$, then the resulted photon can interact with the cristal thus an other electron can be emitted with the following energy:

$$E_{e_2}=E_p-E_{max}$$

\subsubsection{Position-Electron interaction}
During disintegration, $\beta^+$ particle can be emitted, These 
particles are, in fact, positron that can interact with electrons in 
the scintillator to create 2 gamma photons, which has both an energy of $511$ [Kev].
$^{22}_{11}$Na is known to emit those kind of particle, which is why we have a typical peak for this energy level.

\subsubsection{X rearrangement}
When a photoelectric effect occurs and the energy of the $\gamma$ is closed to "33.17 keV" ( which is the energy corresponding to the damping energy). The interaction between the photon and the face of the cristal is highly likely, thus a X photon is likely to escape. The typical peak corresponding is called the "Escaping peak" which is given by:

$$E=E_{\gamma}-E_{X}$$

\subsubsection{Other interactions}
\begin{itemize}
\item It can happen that the X ray is reabsorb by a Compton effect.
\item Background noise coming from surrounding photons of the scintillateur.
\item Energy resolution of the detector, which is the ability to discriminate two photons of similar energy level.
\end{itemize}

\newpage
\subsection{Figures}
\begin{comment}
\subsubsection{Co 57}
\begin{figure}[h!]
	\begin{center}
	\includegraphics[width=1.0\linewidth,angle=0]{./figures/Co57.pdf}
	\caption{} \label{fig:Co57}
	\end{center}
\end{figure}


\newpage
\end{comment}
\subsubsection{Co 60}
\begin{figure}[h!]
	\begin{center}
	\includegraphics[width=1.0\linewidth,angle=0]{./figures/Co60.pdf}
	\caption{$_{27}^{60}$Co spectrum} \label{fig:Co60}
	\end{center}
\end{figure}

$_{27}^{60}$Co is desintegrating into $_{28}^{60}$Ni which gives 2 gamma photons, $\gamma_1,\gamma_2$.

\begin{itemize}
\item Peak \#1, from the collimator lead, E=69.07 [KeV]
\item Peak \#2, back-scattering with $\gamma_1,\gamma_2$ confounded, E=215.84 [KeV]
\item Peak \#3, End of Compton effect $\gamma_1,\gamma_2$ confounded, E=892.16 [KeV]
\item Peak \#4, Photoelectric peak from $\gamma_1$, E=1159.82 [KeV]
\item Peak \#5, Photoelectric peak from $\gamma_2$, E=1318.10 [KeV]
\end{itemize}


\newpage
\subsubsection{Cs 137}
\begin{figure}[h!]
	\begin{center}
	\includegraphics[width=1.0\linewidth,angle=0]{./figures/Cs137.pdf}
	\caption{$_{55}^{137}$Cs spectrum} \label{fig:Cs137}
	\end{center}
\end{figure}

$_{55}^{137}$Cs is desintegrating into $_{56}^{137}$Ba which gives 1 gamma photon, $\gamma$

\begin{itemize}
\item Peak \#1, from the collimator lead, E=64.23 [KeV]
\item Peak \#2, back-scattering, E=182.49 [KeV]
\item Peak \#3, End of Compton effect, E=408.79 [KeV]
\item Peak \#4, Photoelectric peak from $\gamma$, E=642.39 [KeV]
\end{itemize}


\newpage
\subsubsection{Na 22}
\begin{figure}[h!]
	\begin{center}
	\includegraphics[width=1.0\linewidth,angle=0]{./figures/Na22.pdf}
	\caption{$_{11}^{22}$Na spectrum} \label{fig:Na22}
	\end{center}
\end{figure}

$_{11}^{22}$Na is desintegrating into $_{10}^{22}$Ne which gives 1 gamma photon, $\gamma$.

\begin{itemize}
\item Peak \#1, from the collimator lead, E=67.54 [KeV]
\item Peak \#2, Annihlation of $e^+e^-$, E=496.17 [KeV]
\item Peak \#3, Photoelectric peak from $\gamma$, E=1254.71 [KeV]
\end{itemize}


\newpage
\subsubsection{Pb 120}
\begin{figure}[h!]
	\begin{center}
	\includegraphics[width=1.0\linewidth,angle=0]{./figures/Pb120.pdf}
	\caption{$_{82}^{210}$Pb spectrum} \label{fig:Pb120}
	\end{center}
\end{figure}

$_{82}^{210}$Pb is desintegrating into $_{83}^{210}$Bi which gives 2 gamma photons, $\gamma$.

\begin{itemize}
\item Peak \#1, from photoelectric effect, E=35.47 [KeV]
\end{itemize}

\newpage
\subsection{Spectrum Analysis}
By reporting the peak of the different spectrum and knowing the energy level of this peak, we can obtain a linear curve showing that the detection configuration has a linear behaviour and thus we can guess the different energy level of the Hf181:


\begin{figure}[h!]
	\begin{center}
	\includegraphics[width=1.0\linewidth,angle=0]{./figures/Spectrumcal.pdf}
	\caption{Plot of the linear function fit to the calibration process, one can see a linear dependance between an energy level and the channel, the slope is approximatley 2.93 [KeV/Channel].} \label{fig:calibrate}
	\end{center}
\end{figure}

\newpage
Finally, one can verify the dependance of resolution $\sigma_E/E$ function of $\frac{1}{\sqrt{E}}$.
\begin{figure}[h!]
	\begin{center}
	\includegraphics[width=1.0\linewidth,angle=0]{./figures/Resolution.pdf}
	\caption{Plot of the system resolution, $\sigma_E/E$ function of $\frac{1}{\sqrt{E}}$, the fit is a linear function.} \label{fig:resolution}
	\end{center}
\end{figure}

\newpage
\subsection{Hf 181}

\begin{figure}[h!]
	\begin{center}
	\includegraphics[width=1.0\linewidth,angle=0]{./figures/Hf181.pdf}
	\caption{$_{72}^{181}$Hf spectrum} \label{fig:Hf181}
	\end{center}
\end{figure}

$_{72}^{181}$Na is desintegrating into $_{72}^{181}$Ta which gives 1 gamma photon, $\gamma$.

\begin{itemize}
\item Peak \#1, E=115.31 [KeV]
\item Peak \#2, E=320.15 [KeV]
\item Peak \#3, E=461.24 [KeV]
\end{itemize}

\newpage
\section{Poisson Distribution}

This section shows the statistic behaviour of nuclear particles detection. We can use a Poissonian distribution to describe this behaviour.
The experiment is made from 512 measures, One measure lasts $\Delta t$, thus the experiment takes 512*$\Delta t$ s to be done.

The mean $\overline{N}$ of the sample is given by: $$\overline{N}=\frac{\Sigma N_i}{512}$$
The deviation is given by: $$\sigma_{\overline{N}}=\sqrt{\frac{1}{512}\Sigma^{\infty}_{N=0} (N-\overline{N})^2 f(N)}$$
Also, the theorical probability of Poisson distribution is given by: $$p(N)=\overline{N}^N \frac{e^{\overline{N}}}{N!}$$

Finally, as we know that the Poisson law tends to a normal law when $N_i$ goes to infinite, we can apply the $\chi^2$ test.

$$T^2=\Sigma^{14}_{i=1}{\left(\frac{f(N_i)-n*p(N_i)}{\sqrt{n*p(N_i)}}\right)^2}$$

The following histogram shows the theorical and the experimental distribution for the experiments with different $\Delta t$ of time integration.

Several measures have been made by varying $\Delta t$ which are shown in the table~\ref{tab:poisson}.

\begin{figure}[h!]
	\centering

	\subfloat{\label{figur:1}\includegraphics[width=0.5\linewidth,angle=0]{./figures/Poisson_5ms.pdf}}
	\subfloat{\label{figur:2}\includegraphics[width=0.5\linewidth,angle=0]{./figures/Poisson_6ms.pdf}}
	\\
	\subfloat{\label{figur:3}\includegraphics[width=0.5\linewidth,angle=0]{./figures/Poisson_7ms.pdf}}
	\subfloat{\label{figur:4}\includegraphics[width=0.5\linewidth,angle=0]{./figures/Poisson_8ms.pdf}}
	\\
	\subfloat{\label{figur:5}\includegraphics[width=0.5\linewidth,angle=0]{./figures/Poisson_9ms.pdf}}
	\subfloat{\label{figur:6}\includegraphics[width=0.5\linewidth,angle=0]{./figures/Poisson_10ms.pdf}}
	\\
	\subfloat{\label{figur:5}\includegraphics[width=0.5\linewidth,angle=0]{./figures/Poisson_15ms.pdf}}
	\subfloat{\label{figur:6}\includegraphics[width=0.5\linewidth,angle=0]{./figures/Poisson_20ms.pdf}}
	\label{figur}\caption{Histogram of different $\Delta t$ for Poissonian distribution behaviour of nuclear phenomena.}
\end{figure}

\begin{table}[h!]
\centering
	\begin{tabular}{|p{2cm}|p{2cm}|p{3cm}|c|c|c|c|}
		 \hline
		 Type (time) & Theory ($\overline{N}\pm\sigma$)& Experimental ($\overline{N}\pm\sigma$) & one trial & 100 trials & $\chi^2$  & $\chi^2_{Theory}$\\
		\hline
		 5 ms & $6\pm 2.45$ & $5.94\pm 2.41$ & 0.41 & 0.018 & 6.32 & 7.790\\
		 6 ms & $8\pm 2.82$ & $7.3\pm 2.72$ & 0.37 & 0.016 & 19.67 & 9.312\\
		 7 ms & $9\pm 3.0$ & $8.5\pm 3.03$ & 0.34 & 0.015 & 28.15 & 10.085\\
		 8 ms & $10\pm 3.16$ & $9.82\pm 3.18$ & 0.31 & 0.014 & 19.1 & 11.651\\
		 9 ms & $11\pm 3.31$ & $10.7\pm 3.28$ & 0.30 & 0.013 & 14.12 & 12.443\\
		 10 ms & $12\pm 3.46$ & $12.2\pm 3.54$ & 0.28 & 0.012 & 12.38 & 15.659\\
		 15 ms & $18\pm 4.24$ & $18.1\pm 4.16$ & 0.23 & 0.010 & 24.24 & 20.599\\
		 20 ms & $24\pm 4.89$ & $24.35\pm 5.26$ & 0.20 & 0.008 & 46.2 & 29.051\\
		\hline
	\end{tabular}
	\caption{Summary of values for the Poissonian Distribution}
	\label{tab:poisson}
\end{table}

As we see on the table~\ref{tab:poisson} for different $\Delta t$ experimental results keeps to the theorical one. One can also see that doing 100 trials is more precise than only a long one. Finally, the Chi squared test with the correspond parameter $\nu$ and using a confidence of 90$\%$ isnot concluding. Datas are close but not good enough to have a sufficient confidence for standard physics convention.

\newpage
\section{Attenuation in matter}
We know from the different kind of effect that the gamma rays can be attenuate in the matter when it goes through. The attenuation correspond of a discreasing amount of photons received without lost of energy from the photons.
This phenomena is described by an exponential law:
$$I=I_0e^{-\mu(E,Z)x}=I_0 e^{-\mu_d d}$$
,where $\mu_d=\frac{\mu}{\rho}$ which is the mass attenuation coefficient and $d=\rho x$ is the area screen density.

The figure~\ref{fig:attenuation} shows the experimental results from the attenuation of Cs137 through an aluminium or a lead obstacle with different thickness. The slope of this curves correspond to the attenuation coefficient $\mu$ (c.f table~\ref{tab:attenuation}).
Whereas the $\mu$ values of both samples are clearly different, the $\mu_d$ coefficient seems to be quite similar. One can also see that the lead is more efficient to protect someone from the radiation than aluminium. This is due to the higher density of lead, which corresponds to a higher probability of interaction between a gamma photons and matter.

\begin{figure}[h!]
	\begin{center}
	\includegraphics[width=1.0\linewidth,angle=0]{./figures/Attenuation.pdf}
	\caption{Attenuation fit for both aluminium an lead, the lead seems to attenuate more efficiently than aluminium, therefore its wiser to use lead.} \label{fig:attenuation}
	\end{center}
\end{figure}

\begin{table}[h!]
\centering
	\begin{tabular}{|c|c|c|}
		 \hline
		Material & $\mu$ [$cm^{-1}$] & $\mu_d$ [$cm^2/g$]\\
		\hline
		Pb	& 0.97 & 0.085 \\
		Al	& 0.18 & 0.066 \\
			\hline
	\end{tabular}
	\caption{Resume of the values from lead and aluminium attenuation thus to compare, Pb is more efficient, although, both $\mu_d$ seems quite close.}
	\label{tab:attenuation}
\end{table}

\newpage
\section{Coincidence}

Two detectors (for beta or gamma rays) are placed face to face and a radioactive sample is placed in between. By measuring all the coincidences of detected events by those two detectors, different kind of coincidences are measured : some come from a relationship between the emmited particles, some come from random noise from the cosmic background, and some come from fortuit coincidence. So $m_m$ the measured coincidence rate can be written like this :

\be 
	m_m = m_f + m_c + m_r
\ee 

where $m_f$ represent the rate due to the fortuit collision, $m_c$ the rate due to the cosmic background (noise) and $m_r$ the rate due to real coincidence that is beeing observed. 

\subsection{cosmic background coincidence measurement}


\begin{table}[h!]
\centering
	\begin{tabular}{|c|c|c|c|}
		\hline
			gamma count		&beta count		&coincidence	&time\\
		\hline
			7760			&246			&0				&10327\\
		\hline
	\end{tabular}
	\caption{measurements}
	\label{tab:poisson}
\end{table}


Some measurement were taken without any samples : some particles were detected, but the the coincidence rate was so low that it wasn't possible to get any of those during $1033 s$ ($17 min$). It is so low that it can be negliged. so the measurement for $m_m$ becomes :

\be 
	m_m = m_f + m_r
\ee 

\subsection{fortuit coincidence measurement}

Two independant radioactiv sources are separated by a Pb plate, and a $\beta$ and a $\gamma$ detector are placed in front of each sources. Some measurements are then taken on both of the detectors and the rate of fortuitous coincidences is measured. The rate of detection is dependant to the resolution of the detector : an event is actually represented by a time lapse $\Delta t$ and the coincidence rate depend on the length of that time lapse like this :

\be
	m_f = 2\theta (m_1-m_r) (m_2-m_r)
\ee

where $\theta$ is the sum of the time lapse of each detector, $m_1$ and $m_2$ are the rate of both detectors and $m_r$ is the real coincidence. 

So in this case $m_r$ is set to $0$ so that $\theta$ can be measured.

\be
	2\theta = \frac{m_f}{m_1 m_2}
\ee


% type[*][*]	nb_coups_gamma[*][*]		nb_coups_beta[*][*]		coincidence[*][*]		temps[s][*]
% fortuite		2198562						147272					1304					2444

For the measurement we had $2198562$ gamma rays, $147272$ beta rays with $1304$ coincidences in $2444$ seconds.
That means
% $2\theta=\frac{m_f}{m_1m_2} = (1304/2444) / ( (2198562/2444) * (147272/2444) ) = 9.8428\cdot10^{-6} seconds$
$2\theta=\frac{m_f}{m_1m_2}                                                      = 9.8428\cdot10^{-6} seconds$








% !!!!! GRAPH : $m_f$ en fonction de $m_1*m_2$ !!!!!
% CALCULER LA PENTE, C'EST EGALE A 2THETA


\subsection{tentale 181 activity, period and age of the sample}


\subsubsection{activity}

	% \begin{figure}[h!]
	% 	\begin{center}
	% 	\includegraphics[width=1.0\linewidth,angle=0]{./figures/2theta.pdf}
	% 	\caption{Plot of $	m_f = 2\theta m_1 m_2$ with a linear fit} \label{fig:2theta}
	% 	\end{center}
	% \end{figure}

	The activity of the tentale 181 sample we had can be measured with this relation :

	\be
		A = \frac{m_1 m_2}{2 m_r} = \frac{m_1 m_2}{2(m_m - 2\theta m_1 m_2)}
	\ee

	where $m_1$ and $m_2$ are the measurements got in the last part and $m_r$ is the real coincidence discussed before, that value can be estimated to $m_m - 2\theta m_1 m_2$.


	\begin{table}[h!]
	\centering
		\begin{tabular}{|c|c|c|c|c|}
			\hline
				beta count	&gamma count	&coincidence	&time[s] 	&$2\theta$[s]\\
			\hline
				44324		&1052777		&104			&895.1		&$9.8\cdot10^{-6}$\\
			\hline
		\end{tabular}
		\caption{measurements}
		\label{tab:poisson}
	\end{table}

	% ((44324/895.1) * (1052777/895.1))/(2*104/895.1 - 2*9.8*10^(-6) * (44324/895.1) * (1052777/895.1))
	$A = -6.4061e+04 $



	% The fit given in the graph \ref{fig:2theta} show a value for $2A$ of $1.29 \cdot 10^{-6} \pm 1.6\cdot 10^{-7} (12.5\%) seconds$ and the value at the origin is almost $0$.

	

	% !!! ON A QQCH COMME $A = m1m2 / m12 = 4.3259 10^4$ QUI EST OK SELON L'ASSISSTANTE
	% !!! AJOUTER TABLE OU GRAPHE DES DATAS RECOLTES


\subsubsection{period}

	By measuring the coincidence with a delay, using the relation :

	\be
		p(t)= \lambda e^{-\frac{t}{\tau}}
	\ee

	(where $p$ represent the distribution function that represent the lifetime of the observed particle, $\lambda$ a factor for normalization, $t$ the delay and $\tau$ the period of the observed particle) we can measure and calculate the $\tau$ parameter. 


	\begin{figure}[h!]
		\begin{center}
		\includegraphics[width=1.0\linewidth,angle=0]{./figures/coincidence.pdf}
		\caption{Plot of $	m_f = 2\theta m_1 m_2$ with a linear fit} \label{fig:2theta}
		\end{center}
	\end{figure}


	% $\tau = 1/0.00994745$
	$\tau = 1.0348 \cdot 10^8$ seconds which is $3.27915$ years.
	% 0.00966398       +/- 0.0005028    (5.203%)

\subsubsection{age}

	Knowing that the activity of the sample was at the begining $A_0 = 3.7 10^6 Bq$, with the relation :
	\be
		t=ln(A0/A)\tau
	\ee
	we can estimate the age of the sample, since we have the value of $\tau$ and $A$, we can evaluate the sample to be from QPIOPIQUTPIEUTPIEUPITUPIEQTUPEQITUPIEQUTEPITUEQPITUEQIQEUPITUIPUQPIUPQUTEIP 

	% 1990.8=1991

	% $ t=ln(A0/A)\tau $

	% $ \tau(CO60)=5.3 ans $

	% $ A0=3.7 10^6 $





\newpage
\section{Conclusion}
The results are quite correct, the spectrum analysis gives the ability to calibrate and measure some unknown spectrum as the Hafmium one. One can also see that the nuclear decay phenomena follows a Poissonian distribution and the gamma rays can be efficiently stopped with lead unlinke aluminium. Finally, by knowing all these parameters, the coincidence experiments has been driven, results werenot so correct but still interesting. In the overall, one can say that Nuclear physics is highly interesting but also complicated, although its applications are large, like in medecine or in energy production.




%Reference
%\begin{thebibliography}{99}
%\end{thebibliography}

\end{document}
