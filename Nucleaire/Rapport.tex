\documentclass[a4paper,12pt,oneside]{article}

\usepackage{subfig}
\usepackage{graphicx}
\usepackage{verbatim}
\usepackage{amsmath}
\usepackage[english]{babel}
\usepackage[utf8]{inputenc} 
\usepackage[colorlinks,bookmarks=false,linkcolor=blue,urlcolor=blue]{hyperref}
\usepackage{booktabs}

\paperheight=297mm
\paperwidth=210mm

\setlength{\textheight}{235mm}
\setlength{\topmargin}{-1.2cm}
%\setlength{\footskip}{5mm}
\setlength{\textwidth}{15cm}
\setlength{\oddsidemargin}{0.56cm}
\setlength{\evensidemargin}{0.56cm}

\pagestyle{plain}


\def \be {\begin{equation}}
\def \ee {\end{equation}}
\def \dd  {{\rm d}}

\newcommand{\mail}[1]{{\href{mailto:#1}{#1}}}
\newcommand{\ftplink}[1]{{\href{ftp://#1}{#1}}}


\begin{document}

\title{}
\author{Laurent Rohrbasser \& Tim Tuuva}

\maketitle
\tableofcontents
\baselineskip=16pt
\parindent=15pt
\parskip=5pt

\begin{abstract}
%Résumé de l'expérience, on fait des tps sur le chaos, rappeler vite fait 
%dire le but de ces manips, qu'est ce qu'on veut?
\end{abstract}

\section{Introduction}
%expliqué les enjeux théoriques sur le chaos.
%domaine très théoriques
%peu de compréhension sur le sujet ( pourquoi???)

\section{Spectrometry $\gamma$}
\subsection{Theory}
Radioactive sources generates one or multiple gamma photons of various 
energy level. However, these photons cannot directly interact with the
 detector by a photoelectric effect due to their high energy level. 
%TODO iodure de sodium
This is why, "iodure de sodium" cristal is used, thus to make the 
photons react with the cristal which generates electrons, this 
process generates new photons which will interact with electrons. 
This cristal realises, therefore, a chain reaction that amplifies the 
number of photons, which are less powerful than the original one, 
thus it can be detected on the photocathode. The measured tension 
depends on the energy level of the initial gamma photons.

There are several kind of phenomena that can happen on the gamma photons.
\subsubsection{Photoelectric effect}
This effect occurs when a photon has enough energy to free an electron from its atom. The kinetic energy of the ejected electron is given by:

$$T=E_p - E_l$$

, where $E_l$ is the link energy between the electron and the atom.

\subsubsection{Compton effect}
This effect occurs when a photon gives a part of its momentum to an 
electron as they interact. The electron will follow a direction given 
by the angle $\phi$ and a new photon will be emitted with an angle $\theta$.
The kinetic energy of the electron is given by:

$$T=\frac{E_p}{1+\frac{m_0c^2}{E_p(1-\cos(\theta))}}$$

The Compton effect creates a flat pattern on the energy spectrum, it ends at $E_{max}=XXX$%TODO,

One can notice that if the angle of scattering is $\theta=180^{\circ}$, then the resulted photon can interact with the cristal thus an other electron can be emitted with the following energy:

$$E_{e_2}=E_p-E_{max}$$

\subsubsection{Position-Electron interaction}
During disintegration, $\beta^+$ particle can be emitted, These 
particles are, in fact, positron that can interact with electrons in 
%TODO  change scintilleur
the "scintillateur" to create 2 gamma photons, which has both an energy of $511$ [Kev].
$^{22}_{11}$Na is known to emit those kind of particle, which is why we have a typical peak for this energy level.

\subsubsection{X rearrangement}%TODO rearrangement
When a photoelectric effect occurs and the energy of the $\gamma$ is closed to "33.17 keV" ( which is the energy corresponding to the damping energy). The interaction between the photon and the face of the cristal is highly likely, thus a X photon is likely to escape. The typical peak corresponding is called the "Escaping peak" which is given by:

$$E=E_{\gamma}-E_{X}$$

\subsubsection{Other interactions}
\begin{itemize}
\item It can happen that the X ray is reabsorb by a Compton effect.
\item Background noise coming from surrounding photons of the scintillateur.
\item Energy resolution of the detector, which is the ability to discriminate two photons of similar energy level.
\end{itemize}

\subsection{Figures}

\begin{figure}[H]
  \begin{center}
  \includegraphics[width=1.0\linewidth,angle=0]{./figures/Co57.pdf}
  \caption{} \label{fig:Co57}
  \end{center}
\end{figure}

\begin{figure}[H]
  \begin{center}
  \includegraphics[width=1.0\linewidth,angle=0]{./figures/Co60.pdf}
  \caption{} \label{fig:Co60}
  \end{center}
\end{figure}

\begin{figure}[H]
  \begin{center}
  \includegraphics[width=1.0\linewidth,angle=0]{./figures/Cs137.pdf}
  \caption{} \label{fig:Cs137}
  \end{center}
\end{figure}

\begin{figure}[H]
  \begin{center}
  \includegraphics[width=1.0\linewidth,angle=0]{./figures/Na22.pdf}
  \caption{} \label{fig:Na22}
  \end{center}
\end{figure}

\begin{figure}[H]
  \begin{center}
  \includegraphics[width=1.0\linewidth,angle=0]{./figures/Pb120.pdf}
  \caption{} \label{fig:Pb120}
  \end{center}
\end{figure}

\subsection{Spectrum Analysis}
By reporting the peak of the different spectrum and knowing the energy level of this peak, we can obtain a linear curve showing that the detection configuration has a linear behaviour and thus we can guess the different energy level of the Hf181:

%TODO caption and errors
\begin{figure}[H]
  \begin{center}
  \includegraphics[width=1.0\linewidth,angle=0]{./figures/Spectrumcal.pdf}
  \caption{} \label{fig:calibrate}
  \end{center}
\end{figure}

Finally, one can verify the dependance of resolution $\sigma_E/E$ function of $\frac{1}{\sqrt{E}}$.
\begin{figure}[H]
  \begin{center}
  \includegraphics[width=1.0\linewidth,angle=0]{./figures/Resolution.pdf}
  \caption{} \label{fig:resolution}
  \end{center}
\end{figure}

\subsection{Hf 181}
%TODO ANALYSIS
\begin{figure}[H]
  \begin{center}
  \includegraphics[width=1.0\linewidth,angle=0]{./figures/Hf181.pdf}
  \caption{} \label{fig:Hf181}
  \end{center}
\end{figure}

\section{Poisson Distribution}

This section shows the statistic behaviour of nuclear particles detection. We can use a Poissonian distribution to describe this behaviour.
The experiment is made from 512 measures, One measure lasts $\Delta t$, thus the experiment takes 512*$\Delta t$ s to be done.

The mean $\overline{N}$ of the sample is given by: $$\overline{N}=\frac{\Sigma N_i}{512}$$
The deviation is given by: $$\sigma_{\overline{N}}=\sqrt{\frac{1}{512}\Sigma^{\infty}_{N=0} (N-\overline{N})^2 f(N)}$$
Also, the theorical probability of Poisson distribution is given by: $$p(N)=\overline{N}^N \frac{e^{\overline{N}}}{N!}$$
%TODO complete

Finally, as we know that the Poisson law tends to a normal law when $N_i$ goes to infinite, we can apply the $\chi^2$ test.

$$T^2=\Sigma^{14}_{i=1}{\left(\frac{f(N_i)-n*p(N_i)}{\sqrt{n*p(N_i)}}\right)^2}$$

The following histogram shows the theorical and the experimental distribution for the experiments with different $\Delta t$ of time integration.

Several measures have been made by varying $\Delta t$ which are shown in the table~\ref{tab:poisson}.

%TODO captions
\begin{figure}[H]
  \centering

  \subfloat[Subcaption 1]{\label{figur:1}\includegraphics[width=0.5\linewidth,angle=0]{./figures/Poisson_5ms.pdf}}
  \subfloat[Subcaption 2]{\label{figur:2}\includegraphics[width=0.5\linewidth,angle=0]{./figures/Poisson_6ms.pdf}}
  \\
  \subfloat[Subcaption 3]{\label{figur:3}\includegraphics[width=0.5\linewidth,angle=0]{./figures/Poisson_7ms.pdf}}
  \subfloat[Subcaption 4]{\label{figur:4}\includegraphics[width=0.5\linewidth,angle=0]{./figures/Poisson_8ms.pdf}}
  \\
  \subfloat[Subcaption 5]{\label{figur:5}\includegraphics[width=0.5\linewidth,angle=0]{./figures/Poisson_9ms.pdf}}
  \subfloat[Subcaption 6]{\label{figur:6}\includegraphics[width=0.5\linewidth,angle=0]{./figures/Poisson_10ms.pdf}}
  \\
  \subfloat[Subcaption 5]{\label{figur:5}\includegraphics[width=0.5\linewidth,angle=0]{./figures/Poisson_15ms.pdf}}
  \subfloat[Subcaption 6]{\label{figur:6}\includegraphics[width=0.5\linewidth,angle=0]{./figures/Poisson_20ms.pdf}}
  \label{figur}\caption{Histogram}
\end{figure}

\begin{table}[H]
\centering
	\begin{tabular}{|c|c|c|c|c|c|}
	   \hline
	   Type (time) & Theory ($\overline{N}\pm\sigma$)& Experimental ($\overline{N}\pm\sigma$) & one trial & 100 trials & $\chi^2$ \\
		\hline
	   5 ms & $6\pm 2.45$ & $5.94\pm 2.41$ & 0.41 & 0.018 & 6.32\\
	   6 ms & $8\pm 2.82$ & $7.3\pm 2.72$ & 0.37 & 0.016 & 19.67\\
	   7 ms & $9\pm 3.0$ & $8.5\pm 3.03$ & 0.34 & 0.015 & 28.15\\
	   8 ms & $10\pm 3.16$ & $9.82\pm 3.18$ & 0.31 & 0.014 & 19.1\\
	   9 ms & $11\pm 3.31$ & $10.7\pm 3.28$ & 0.30 & 0.013 & 14.12\\
	   10 ms & $12\pm 3.46$ & $12.2\pm 3.54$ & 0.28 & 0.012 & 12.38\\
	   15 ms & $18\pm 4.24$ & $18.1\pm 4.16$ & 0.23 & 0.010 & 24.24\\
	   20 ms & $24\pm 4.89$ & $24.35\pm 5.26$ & 0.20 & 0.008 & 46.2\\
		\hline
	\end{tabular}
	\caption{Summary of values for the Poissonian Distribution}
	\label{tab:poisson}
\end{table}

%TODO ANALYSIS OF TABLE

\section{Attenuation in matter}%le plus important
We know from the different kind of effect that the gamma rays can be attenuate in the matter when it goes through. The attenuation correspond of a discreasing amount of photons received without lost of energy from the photons.
This phenomena is described by an exponential law:
$$I=I_0e^{-\mu(E,Z)x}=I_0 e^{-\mu_d d}$$
,where $\mu_d=\frac{\mu}{\rho}$ which is the mass attenuation coefficient and $d=\rho x$ is the area screen density.

The figure~\ref{fig:attenuation} shows the experimental results from the attenuation of Cs137 through an aluminium or a lead obstacle with different thickness. The slope of this curves correspond to the attenuation coefficient $\mu$ (c.f table~\ref{tab:attenuation}).
Whereas the $\mu$ values of both samples are clearly different, the $\mu_d$ coefficient seems to be quite similar. One can also see that the lead is more efficient to protect someone from the radiation than aluminium. This is due to the higher density of lead, which corresponds to a higher probability of interaction between a gamma photons and matter.%TODO energy DECAY table D6 D7???

%COMPLETE

%TODO
\begin{figure}[H]
  \begin{center}
  \includegraphics[width=1.0\linewidth,angle=0]{./figures/Attenuation.pdf}
  \caption{} \label{fig:attenuation}
  \end{center}
\end{figure}

\begin{table}[H]
\centering
	\begin{tabular}{|c|c|c|}
	   \hline
		Material & $\mu$ [$cm^{-1}$] & $\mu_d$ [$cm^2/g$]\\
		\hline
		Pb	& 0.97 & 0.085 \\
		Al	& 0.18 & 0.066 \\
	   	\hline
	\end{tabular}
	\caption{TODO}
	\label{tab:attenuation}
\end{table}

\subsection{Interprétation}
%Qu'est ce que les résultats nous permettent de conclure?
%Est ce que cela nous aide pour nôtre but?

\subsection{Discussions}
%Validité de nos résultats
%S'il y a des erreurs, d'ou viennent elle!
%Avons nous atteind les buts du TP?
\section{Conclusion}

%Résumer du rapport
%Ouverture





%Reference
\begin{thebibliography}{99}
\end{thebibliography}

\end{document}
