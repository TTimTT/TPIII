\documentclass[a4paper,12pt,oneside]{article}

\usepackage{graphicx}
\usepackage{verbatim}
\usepackage{amsmath}
\usepackage[english]{babel}
\usepackage[colorlinks,bookmarks=false,linkcolor=blue,urlcolor=blue]{hyperref}
\usepackage{booktabs}

\paperheight=297mm
\paperwidth=210mm

\setlength{\textheight}{235mm}
\setlength{\topmargin}{-1.2cm}
%\setlength{\footskip}{5mm}
\setlength{\textwidth}{15cm}
\setlength{\oddsidemargin}{0.56cm}
\setlength{\evensidemargin}{0.56cm}

\pagestyle{plain}


\def \be {\begin{equation}}
\def \ee {\end{equation}}
\def \dd  {{\rm d}}

\newcommand{\mail}[1]{{\href{mailto:#1}{#1}}}
\newcommand{\ftplink}[1]{{\href{ftp://#1}{#1}}}


\begin{document}

\title{}
\author{Laurent Rohrbasser \& Tim Tuuva}

\maketitle
\tableofcontents
\baselineskip=16pt
\parindent=15pt
\parskip=5pt

\begin{abstract}
%Résumé de l'expérience, on fait des tps sur le chaos, rappeler vite fait 
%dire le but de ces manips, qu'est ce qu'on veut?
\end{abstract}

\section{Introduction}
%expliqué les enjeux théoriques sur le chaos.
%domaine très théoriques
%peu de compréhension sur le sujet ( pourquoi???)

\section{Theory}
\subsection{Spectrometry $\gamma$}
Radioactive sources generates one or multiple gamma photons of various 
energy level. However, these photons cannot directly interact with the
 detector by a photoelectric effect due to their high energy level. 
%TODO iodure de sodium
This is why, "iodure de sodium" cristal is used, thus to make the 
photons react with the cristal which generates electrons, this 
process generates new photons which will interact with electrons. 
This cristal realises, therefore, a chain reaction that amplifies the 
number of photons, which are less powerful than the original one, 
thus it can be detected on the photocathode. The measured tension 
depends on the energy level of the initial gamma photons.

There are several kind of phenomena that can happen on the gamma photons.
\subsubsection{Photoelectric effect}
This effect occurs when a photon has enough energy to free an electron from its atom. The kinetic energy of the ejected electron is given by:

$$T=E_p - E_l$$

, where $E_l$ is the link energy between the electron and the atom.

\subsubsection{Compton effect}
This effect occurs when a photon gives a part of its momentum to an 
electron as they interact. The electron will follow a direction given 
by the angle $\phi$ and a new photon will be emitted with an angle $\theta$.
The kinetic energy of the electron is given by:

$$T=\frac{E_p}{1+\frac{m_0c^2}{E_p(1-\cos(\theta))}}$$

The Compton effect creates a flat pattern on the energy spectrum, it ends at $E_{max}=XXX$%TODO,

One can notice that if the angle of scattering is $\theta=180^{\circ}$, then the resulted photon can interact with the cristal thus an other electron can be emitted with the following energy:

$$E_{e_2}=E_p-E_{max}$$

\subsubsection{Position-Electron interaction}
During disintegration, $\beta^+$ particle can be emitted, These 
particles are, in fact, positron that can interact with electrons in 
%TODO  change scintilleur
the "scintilleur" to create 2 gamma photons, which has both an energy of $511$ [Kev].
$^{22}_{11}$Na is known to emit those kind of particle, which is why we have a typical peak for this energy level.


\section{Dispositif}
%Description du dispositif expérimental
%Tim's duty


\section{Résultats}
%Exposition des graphes avec les légendes et un peu de texte explicatifs sur comment on les a obtenus.

\begin{figure}[h!]
  \begin{center}
  %\includegraphics[width=0.8\linewidth,angle=0]{}
  \caption{} \label{fig:}
  \end{center}
\end{figure}

\section{Discussion}%le plus important

\subsection{Interprétation}
%Qu'est ce que les résultats nous permettent de conclure?
%Est ce que cela nous aide pour nôtre but?

\subsection{Discussions}
%Validité de nos résultats
%S'il y a des erreurs, d'ou viennent elle!
%Avons nous atteind les buts du TP?
\section{Conclusion}

%Résumer du rapport
%Ouverture





%Reference
\begin{thebibliography}{99}
\end{thebibliography}

\end{document}
